\chapter{Future Work}
ModelDB S+C is a usable end-to-end system with other systems being built on top of it.
However, there are a number of areas for improvement.

\section{Neural Networks}
ModelDB S+C supports special storage and algorithms for linear and tree models. It
may be interesting to also add support for neural network models. This was omitted 
for this thesis because Spark.ML supports only simple multi-layer perceptron models,
which is one kind of neural network. Adding support for more general neural network models,
especially deep neural networks, could be valuable.

\section{Library Agnostic Model Format}
ModelDB S+C stores detailed data about linear and tree models that could be used
to reconstruct them. Consequently, ModelDB Server's database tables could serve as a library agnostic
storage format for machine learning models. It may be interesting to implement classes in
Spark Client and Scikit-learn Client to read ModelDB Server's tables and reconstruct a Spark Transformer or Scikit-learn
Predictor from the contents of the table. This would make it possible, for example, to create a model in Scikit-learn,
store it in ModelDB Server's database, and then read it and use it in Spark.ML. Currently, ModelDB supports
storage of serialized model files, so a model created in Spark.ML can later be read and used in Spark.ML.

\section{Scalability}
ModelDB Server currently runs on a single node, but its design is stateless. Therefore,
it may be interesting to see how ModelDB Server scales to multiple nodes.

\section{New clients, API methods, columns}
It would be useful to create client libraries for other machine learning libraries,
such as those in R. There are a number of interesting operations that could be performed
on the data stored in ModelDB Server, so it may be useful to implement more API methods. Finally,
storing additional data in the columns of ModelDB Server's tables (e.g. descriptive statistics for each DataFrameColumn),
may enable a whole new set of API methods.
