\chapter{Conclusion}
This thesis describes ModelDB Server and ModelDB Spark Client, which together
constitute an end-to-end system for logging operations and models created in
the model building process.

The primitives and Syncable Event abstractions in ModelDB S+C were presented as
building blocks for representing a wide range of operations, such as random splitting
of a DataFrame, grid search cross validation, and creation of pre-processing pipelines.

ModelDB Server provides a number of algorithms, exposed as Thrift API endpoints, which
leverage the aforementioned abstractions to glean useful information, like model rankings and comparisons,
about the model building process. 

ModelDB S+C stores detailed data about logistic regression, linear regression,
decision tree, random forest, and gradient boosted tree models. It also supports storing
serialized models in a filesystem.

ModelDB Spark Client allows automatic logging of machine learning operations performed in Spark.ML,
requiring only minor changes to code. Its client side abstractions, such as ModelDB Syncer and SyncableEvent, 
are general and can be applied to other machine learning libraries (e.g. Scikit-learn).

ModelDB S+C serves as a foundation for the overall ModelDB system, and other systems have been
built on top of it.

Recording the model building process is currently difficult and time-consuming to do. 
ModelDB S+C aims to make this process much easier and utilize
data about the model building process to provide insights for the data scientist.
ModelDB S+C hopes to be a step towards an era of applications that use data about the model
building process to make the model building process easier, more efficient, and less time consuming.
